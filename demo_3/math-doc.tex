\documentclass{article}
\usepackage{amsmath}
\usepackage{latexsym}

\newcommand{\flowsinto}{\leadsto}
\newcommand{\relitivity}{E=mc^2}

\begin{document}
\title{Typesetting Mathematics in \LaTeX}
\author{Dale Alleshouse \\ dealleshouse@sei.cmu.edu}
\maketitle

\section{Introduction}
\LaTeX\ is extremely powerful when it comes to typesetting mathematics. It's one of the core
strengths of this system. Is this working?

\section{Displaying Mathematics}
There are two ways of displaying maths. One is \emph{inline} and the other is \emph{display} format
— in which the whole math sits on its own set of lines.


\subsection{Inline Mode}
We are going to insert a mathematics equation inline here using a pair of \$ signs: $E=mc^2$. As you
can see, the display (such as line spacing) does not get messed up by the mathematics as it does
with word processing software.

\subsection{Display Mode}
We can also display equations in their own set of lines. To do this, we can use the equation
environment. See Equation~\ref{eq:emc} for further details.

\begin{equation}
  \label{eq:emc}
  E=mc^2
\end{equation}

As you can see, \LaTeX\ inserts the equation number automatically. We can refer to it using the
\verb|\ref| command just as we referred to sections, figures and tables. (E.g.
Equation~\ref{eq:emc}.) To get rid of the equation number, simply use the \emph{star variant} of the
equation environment. (For this, you need the \texttt{amsmath} package.)

\begin{equation*}
  E = mc^2
\end{equation*}

Alternatively, we can use the shorthand keys \verb|\[| and \verb|\]|

\[
  E = mc^2
\]

\section{Mathematical Features}
\LaTeX\ has many builtin features and you can get many more easily. Here, we'll see some of these
features:

Addition, subtraction, multiplication and division:

\[
  10 + 9 - 5 \times 6 \div 27
\]

\[
  \frac{10 + 8}{5 \times 8}
\]

Superscripts and subscripts:

\[
  V = C_f - C_c
\]

\[
  {(\sqrt{-shit})}^2
\]

Summation, union, intersection, big-union, integral:

\[
  \sum_{c} C_p C_t
\]

\[
  x \cup y \cap z
\]

\[
  \bigcup_{i=0}^n x_i
\]

\[
  \bigcap_{i=0}^n x_i
\]

\[
  \int_0^2 x^2
\]

Fractions, brackets, square root:

\[
  \left[
    \frac{
      \sum_0^n x^i
    }{
      \int_{25}^{67} y_i \pi
    }
  \right]

\]

\[
  \left(
    \frac{
      \sum_0^n x^i
    }{
      \int_{25}^{67} y_i \pi
    }
  \right)
\]

\[
  \sqrt{x^2} = x
\]

Greek letters:

\[
  \alpha + \beta + \gamma + \Gamma + \theta + \Theta + \pi + \log_{2}{5} + e + \epsilon
\]

Matrices and vectors. For this, you need to include the \texttt{amsmath} package and then use the
\texttt{bmatrix} or \texttt{pmatrix} environment:

\[
  \begin{pmatrix}
    1   &   2 \\
    3   &   4 \\
  \end{pmatrix}
\]
\[
  \begin{bmatrix}
    1   &   2 \\
    3   &   4 \\
  \end{bmatrix}
\]

Accents:

\[
  \hat{x} + \imath + \hat{\sum_{i=0}^{5} x^i} + \dot{x}
\]

See the \texttt{Math} menu in the IDE for other operations. You can refer to ``Short Math Guide for
\LaTeX'' for a lot more examples.

\section{Using Symbols}
You might come across situations where you need to find new symbols. For this, you can refer to the
``The Comprehensive \LaTeX\ Symbols List''. $x \leadsto y$

(Optional) Since this is a long command, we might want to create a shortcut using the
\verb|\newcommand| command in the preamble. This also allows us to later change the symbol without
having to change the equations.

\[
  x \flowsinto y
\]

\[
  \relitivity
\]

\end{document}
