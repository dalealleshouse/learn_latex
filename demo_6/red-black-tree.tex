% Red-black tree
% Author: Madit
\documentclass{article}
\usepackage{tikz}
%%%<
\usepackage{verbatim}
\usepackage[active,tightpage]{preview}
\PreviewEnvironment{tikzpicture}
\setlength{\PreviewBorder}{10pt}%
%%%>
\begin{comment}
  :Title: Red-black tree
  :Tags: Trees;Graphs
  :Author: Madit
  :Slug: red-black-tree

  A red-black tree is a special type of binary tree, used in computer science
  to organize pieces of comparable data, such as text fragments or numbers.
  (Wikipedia)
\end{comment}
\usetikzlibrary{arrows}

\tikzset{
  treenode/.style = {align=center, inner sep=0pt, text centered,
    font=\sffamily},
  arn_n/.style = {treenode, circle, white, font=\sffamily\bfseries, draw=black,
    fill=black, text width=1.5em},% arbre rouge noir, noeud noir
  arn_r/.style = {treenode, circle, red, draw=red,
    text width=1.5em, very thick},% arbre rouge noir, noeud rouge
  arn_x/.style = {treenode, rectangle, draw=black,
    minimum width=0.5em, minimum height=0.5em}% arbre rouge noir, nil
}

\begin{document}
\begin{tikzpicture}[->,>=stealth',level/.style={sibling distance = 5cm/#1,
    level distance = 1.5cm}]
  \node [arn_n] {33}
  child{ node [arn_r] {15}
    child{ node [arn_n] {10}
      child{ node [arn_r] {5} edge from parent node[above left]
        {$x$}} %for a named pointer
      child{ node [arn_x] {}}
    }
    child{ node [arn_n] {20}
      child{ node [arn_r] {18}}
      child{ node [arn_x] {}}
    }
  }
  child{ node [arn_r] {47}
    child{ node [arn_n] {38}
      child{ node [arn_r] {36}}
      child{ node [arn_r] {39}}
    }
    child{ node [arn_n] {51}
      child{ node [arn_r] {49}}
      child{ node [arn_x] {}}
    }
  }
  ;
\end{tikzpicture}
\end{document}
